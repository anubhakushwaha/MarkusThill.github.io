\documentclass[11pt]{article}
\usepackage{mathtools}
\usepackage{amssymb}

\begin{document}

To summarize, we have a number of 
\begin{align}
k \in \mathbb{N_+} \backslash \{ 1\}
\end{align}
sailors (only one sailor is not interesting, so we remove this case from our discussions), the number of coconuts which were collected by the sailors before the night
\begin{align}
n \in \mathbb{N_+}.
\end{align}
and the number of monkeys
\begin{align}
m \in \mathbb{N_+}, \  m < k.
\end{align}

The first sailor who wakes up first divides the number of coconuts by the number of sailors. He gives the remaining coconuts to the monkeys. This results in k shares, which have the size (n-1)/k. The first sailor buries his (stolen) share, leaving behind the following number of coconuts:
\begin{align}
	n_1 = \frac{n-m}{k}(k-1)=(n-m)\frac{k-1}{k} \label{eq:n1}
\end{align}
The second sailor, who wakes up, repeats the procedure of the first one:
\begin{align}
	n_2 &= (n_1-m)\frac{k-1}{k}=\big((n-m)\frac{k-1}{k} -m  \big) \frac{k-1}{k} \\
	&= (n-1) \big(\frac{k-1}{k} \big)^2 - m\frac{k-1}{k}
\end{align}
Then, the third sailor wakes up and does the same again:
\begin{align}
	n_3 &= \Bigg(\bigg((n-m)\frac{k-1}{k}-m\bigg)\frac{k-1}{k} - m\Bigg) \frac{k-1}{k} \\
	&= (n-m) \Big(\frac{k-1}{k}\Big)^3 - m\Big(\frac{k-1}{k}\Big)^2 - m\frac{k-1}{k}
\end{align}
So, for the i-th sailor, with $1 \leq i\leq k$, we find the general recursive rule:
\begin{align}
	n_i = (n_{i-1}-m)\frac{k-1}{k}.
\end{align}

For every sailor, we subtract off one from the previous number $n_{i-1}$ and then multiply by $\frac{k-1}{k}$. This relationship can also be expressed as (for $i>1$):
\begin{align}
	n_i &= (n-m)\bigg(\frac{k-1}{k} \bigg)^i - m\bigg(\frac{k-1}{k} \bigg)^{i-1} - \ldots - m\frac{k-1}{k} \\
	&= (n-m)\bigg(\frac{k-1}{k} \bigg)^i - m\sum_{\substack{j=1 \\ i>1}}^{i-1}\bigg(\frac{k-1}{k} \bigg)^j
\end{align}
For our convenience, we define the initial number of coconuts as $n_0 = n$. Hence, we can write $n_i$ for all $0 \leq i\leq k$ as:

\begin{align}
	n_i=\begin{cases} 
		n, & \mbox{for} \ i=0 \\ 
		(n-m)\bigg(\frac{k-1}{k} \bigg)^i - m\sum\limits_{\substack{j=1 \\ i>1}}^{i-1}\bigg(\frac{k-1}{k} \bigg)^j, & \mbox{else}
	\end{cases}
	\label{eq:iterativeForm}
\end{align}

In order to find an initial amount of coconuts $n$, we have to ensure that all $n_i$ are dividable by $k$ with a remainder of one (which is given to the monkey). Hence, it has to be ensured that:
\begin{align}
	\forall i \in \{ 0, \ldots, k\}: \ n_i \equiv m\mod k
	\label{eq:forallI}
\end{align}
Remember that in the morning, after all $k$ sailors have woken up, the remaining coconuts are divided among all sailors. So, as stated above, also $n_k$ has to have a remainder of one when divided by $k$. For $n_0=n$ it is sufficient to ensure that $n$ is a multiple of $k$ plus $m$. For larger $i$, everything gets slightly more difficult. We have to find an $n$ for Eq. \eqref{eq:iterativeForm} such that Eq. \eqref{eq:forallI} is satisfied. To do so, let us first simplify the complicated expression (for now assuming $i>1$) in Eq. \eqref{eq:iterativeForm} using the geometric series specified in the Appendix XXXXX:
\begin{align}
	& \ \ (n-m)\bigg(\frac{k-1}{k} \bigg)^i - m\sum\limits_{\substack{j=1 \\ i>1}}^{i-1}\bigg(\frac{k-1}{k} \bigg)^j \label{eq:probWithSum} \\
	&= (n-m)\bigg(\frac{k-1}{k} \bigg)^i - m\frac{ \Big(\frac{k-1}{k}\Big)^i - \frac{k-1}{k}}{\frac{k-1}{k} - 1} \\
	&= (n-m)\bigg(\frac{k-1}{k} \bigg)^i - m\frac{ \Big(\frac{k-1}{k}\Big)^i - \frac{k-1}{k}}{-\frac{1}{k}} \\
	&= (n-m)\bigg(\frac{k-1}{k} \bigg)^i + mk \bigg(\frac{k-1}{k} \bigg)^i-mk+m \\
	&= (n-m+mk)\bigg(\frac{k-1}{k} \bigg)^i -mk+m. \label{eq:iterativeSimple}
\end{align}
Now we can try to find an $n$ that suffices Eq. \eqref{eq:forallI}:
\begin{align}
(n-m+mk)\bigg(\frac{k-1}{k} \bigg)^i -mk+m &\equiv m \mod k \\
(n-m+mk)\bigg(\frac{k-1}{k} \bigg)^i -mk &\equiv 0 \mod k \\
(n-m+mk)\bigg(\frac{k-1}{k} \bigg)^i &\equiv 0 \mod k \label{eq:modulo}
\end{align}
Note that we cannot get rid of the term $mk$, since the power of the fraction is smaller than one and not a natural number. The left-hand side of Eq. \eqref{eq:modulo} should hence be a multiple of $k$. This can be expressed as:
\begin{align}
	(n-m+mk)\bigg(\frac{k-1}{k} \bigg)^i = r\cdot k, 
\end{align}
where for now we choose $r \in \mathbb{Z}$ and then solve for $n$:
\begin{align}
	(n-m+mk)\bigg(\frac{k-1}{k} \bigg)^i &= r\cdot k \\
	n-m+mk &= r\cdot k \bigg(\frac{k}{k-1} \bigg)^i \\
	n &= r\cdot k \bigg(\frac{k}{k-1} \bigg)^i - mk + m \\
	n &= r\cdot \frac{k^{k+1}}{(k-1)^i} - m(k -1) \label{eq:firstN}.
\end{align}
Although we can now compute different values for $n$ according to \eqref{eq:firstN} which will suffice Eq. \eqref{eq:modulo}, we can observe a problem: For example, with $i=k=3$ and $m=r=1$ we receive a value of $n=8.125$, which -- although it suffices Eq. \eqref{eq:modulo} -- is not a natural number. This is due to the fraction in Eq. \eqref{eq:firstN}. However, we show in Appendix XYZABC that numerator and denominator are relatively prime to each other, hence, the fraction cannot be reduced. So, in order to obtain a natural number for $n$, we can drag the denominator to $r$ and assume that $r$ is a multiple of it by replacing $r/(k-1)^i$ with $q \in \mathbb{N_+}$:
\begin{align}
	n &= r\cdot \frac{k^{k+1}}{(k-1)^i} - m(k -1) \\
	&= \frac{r}{(k-1)^i} \cdot k^{k+1} - m(k -1) \\
	&= q \cdot k^{k+1} - m(k -1). \label{eq:finalSolution}
\end{align}
Note that in Eq. \eqref{eq:finalSolution} we have an expression which is independent of $i$ and hence fulfills Eq. \eqref{eq:forallI} for all $i \in \{2, \ldots, k \}$.


You might have noticed that we actually only solved the problem for $i>1$, since the sum in Eq. \eqref{eq:probWithSum} is only evaluated for $i>1$. However, Eq. \eqref{eq:forallI} demands for all $i \in \{0, \ldots, k \}$ -- so also for $i=0$ and $i=1$ -- that $n_i \equiv m\mod k$. We can directly see that Eq. \eqref{eq:finalSolution} is also a valid solution for the case $i=0$ and we can furthermore show that also the case $i=1$ specified in Eq. \eqref{eq:n1} is only a special case of Eq. \eqref{eq:firstN}, with:
\begin{align}
	(n-m)\frac{k-1}{k} &\equiv m\mod k \\
	\Rightarrow (n-m)\frac{k-1}{k} &= s \cdot k + m \\
	n &= s \cdot k\frac{k}{k-1} + m\frac{k}{k-1} + m \\
	&= \frac{sk^2+mk}{k-1} + m \label{eq:casen1}
\end{align}
and:
\begin{align}
	\frac{sk^2+mk}{k-1} + m &= r\cdot \frac{k^{k+1}}{(k-1)^1} - mk + m \\
	sk^2+mk &= r k^{k+1} - mk(k-1) \\
	sk +m &= rk^k - mk + m \\
	s &= rk^{k-1} - m \in \mathbb{N}. 
\end{align}
Hence, if we insert $s$ into Eq. \eqref{eq:casen1} we obtain Eq. \eqref{eq:firstN} which then leads to the general solution in Eq. \eqref{eq:finalSolution}.

Summary

Appendix

Geometric Series 
In this appendix we briefly mention several properties of the geometric series which are required for the derivations in other dependencies.
For $q \neq 1$ and $j<N$ the geometric series can be derived as:
\begin{equation}
\begin{split}
S = q^{j}+q^{j+1}+ \cdots + q^{N-1} \\
qS = q^{j+1} + q^{j+2} + \cdots + q^N \\
qS - S = q^N - q^j \\
S(q-1) = q^N - q^j \\
S = \frac{q^N - q^j}{q-1} = \frac{q^j-q^N}{1-q}
\end{split}
\label{eq:geometricSeries1}
\end{equation}
More generally, the geometric series can be written as:
\begin{equation}
\begin{split}
a_0 \sum_{i=j}^{N-1} {q^i} = a_0 \frac{q^N - q^j}{q-1}
\end{split}
\label{eq:geometricSeries2}
\end{equation}

Special cases

If the series starts with $i=0$ we retrieve:
\begin{equation}
\begin{split}
a_0 \sum_{i=0}^{N-1} {q^i} = a_0 \frac{1- q^N}{1-q}
\end{split}
\label{eq:geometricSeries2a}
\end{equation}
For $q=1$ we obtain:
\begin{equation}
\begin{split}
a_0 \sum_{i=j}^{N-1} {1^i} = a_0 (N-j)
\end{split}
\label{eq:geometricSeries3}
\end{equation}
For an infinite series with $N=\infty$, convergence is achieved for values $|q|<1$:
\begin{equation}
\begin{split}
a_0 \sum_{i=j}^{\infty} {q^i} = a_0 \frac{q^j}{1-q}
\end{split}
\label{eq:geometricSeries4}
\end{equation}

Powers of neighbored natural numbers are relatively prime

In this appendix we briefly show that two neighbored natural numbers $k$ and $k+1$ are relatively prime, as well as their powers. This can be done with a proof by contradiction: 
Let us assume that there exists a divisor $t>1$ for which:
\begin{align}
k &= q \cdot t \\
k+1 &= r \cdot t
\end{align}
This also implies:
\begin{align}
r > q \\
r - q \geq 1
\end{align}
We can also write:
\begin{align}
q \cdot t + 1 &= r \cdot t \\
 1 &= r \cdot t - q \cdot t \\
 1 &= (r- q) \cdot t  \\
 & \Rightarrow r - q = 1 \wedge t = 1.
\end{align}
Since we required $t>1$, we have a contradiction, which means that the two neighbored numbers $k$ and $k+1$ are relatively prime and a fraction containing these two numbers in the numerator and denominator cannot be reduced. Furthermore, since the prime factorization of $k$ and $k+1$ returns two disjoint sets of primes $P_{k}$ and $P_{k+1}$ ($P_{k} \cap P_{k+1} = \emptyset$), it can also be trivially shown that the powers $k^i$ and $(k+1)^j$, with $i,j \in \mathbb{N_+}$ are also relatively prime.

\end{document}
